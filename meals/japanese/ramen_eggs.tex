% Source: https://www.justonecookbook.com/ramen-egg/

\documentclass[a4paper, 11pt]{../../template/CSJacopoRecipe}

\recipetitle{Ramen Eggs}

\recipedescription{Ramen Eggs (Ajitsuke Tamago or Ajitama) are delicious as a topping for ramen or enjoyed as a snack. Read on to learn how to make these flavorful Japanese soft-boiled eggs at home.}

\recipeserves{4 eggs} % Number of servings, leave command empty (i.e. \recipeserves{}) if not required
\recipepreptime{5 mins} % Preparation time, leave command empty (i.e. \recipepreptime{}) if not required
\recipecookingtime{10 mins} % Cooking time, leave command empty (i.e. \recipecookingtime{}) if not required
\recipedifficulty{Beginner} % Difficulty, leave command empty (i.e. \recipedifficulty{}) if not required

\extramethodinfo{Enjoy the ramen eggs within 3–4 days if your eggs are soft-boiled. If your eggs are hard-boiled, you can keep them in the refrigerator for up to a week.} % leave command empty (i.e. \extramethodinfo{}) if not required

%----------------------------------------------------------------------------------------

\begin{document}
	
	\columnratio{0.32, 0.68} % Set the relative widths of the ingredients and methods columns
	
	\begin{paracol}{2}
		
		\outputrecipeheader
		
		\begin{recipeingredients}
			\ingredient{4 large eggs}
			\ingredient{60mL soy sauce}
			\ingredient{60mL mirin}
			\ingredient{60mL sake}
			\ingredient{1 tsp sugar}
		\end{recipeingredients}
		
		\begin{recipemethods}
			\method{In a small saucepan, combine all the ingredients for the marinade.}
			\method{Bring it to a boil and whisk it a few times to let the sugar dissolve completely. Once boiling, lower the heat and simmer for 1 minute. Turn off the heat. Set aside to cool completely.}
			\method{Cook the eggs in gently boiling water for approx. 6 minutes.}
			\method{Take out the eggs and shock them in cold water for 15 minutes, then peel the eggs.}
			\method{Place the eggs in a plastic bag and add the marinade to the bag. Remove the air from the bag and  seal it. Refrigerate for 12–24 hours (it gets saltier with time).}
		\end{recipemethods}
		
	\end{paracol}
\end{document}