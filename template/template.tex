%%%%%%%%%%%%%%%%%%%%%%%%%%%%%%%%%%%%%%%%%
% Jacopo Recipe
% LaTeX Template
% Version 1.0 (February 15, 2022)
%
% This template originates from:
% https://www.LaTeXTemplates.com
%
% Author:
% Vel (vel@latextemplates.com)
%
% License:
% CC BY-NC-SA 4.0 (https://creativecommons.org/licenses/by-nc-sa/4.0/)
%
%%%%%%%%%%%%%%%%%%%%%%%%%%%%%%%%%%%%%%%%%

%----------------------------------------------------------------------------------------
%	PACKAGES AND OTHER DOCUMENT CONFIGURATIONS
%----------------------------------------------------------------------------------------

\documentclass[
	a4paper, % Paper size, use 'a4paper' for A4 (default) or 'letterpaper' for US letter (you may want to adjust margins afterwards)
	11pt, % Default font size, available sizes are: 8pt, 9pt, 10pt, 11pt, 12pt, 14pt, 17pt and 20pt
]{CSJacopoRecipe}

%---------------------------------------------------------------------------------
%	RECIPE INFORMATION
%---------------------------------------------------------------------------------

\recipetitle{Spaghetti bolognese} % Recipe title, will be made uppercase automatically

\recipedescription{Bolognese sauce, known in Italian as ragú alla Bolognese, is a meat-based sauce originating from Bologna, Italy. In Italian cuisine, it is customarily used to dress ``tagliatelle al ragú" and to prepare ``lasagne alla bolognese".} % Recipe description, leave command empty (i.e. \recipedescription{}) if not required

\recipeserves{4} % Number of servings, leave command empty (i.e. \recipeserves{}) if not required
\recipepreptime{25 mins} % Preparation time, leave command empty (i.e. \recipepreptime{}) if not required
\recipecookingtime{40 mins} % Cooking time, leave command empty (i.e. \recipecookingtime{}) if not required
\recipedifficulty{Beginner} % Difficulty, leave command empty (i.e. \recipedifficulty{}) if not required

\extramethodinfo{You can freeze this bolognese sauce for up to 2 months. Cool to room temperature, then place individual portions or whole quantity in airtight containers or freezer bags and expel air. Label, date and freeze. Place in the fridge overnight to thaw.} % Extra information, tips, variations, storage instructions, etc to appear under the methods, leave command empty (i.e. \extramethodinfo{}) if not required

%----------------------------------------------------------------------------------------

\begin{document}

\columnratio{0.32, 0.68} % Set the relative widths of the ingredients and methods columns, change these values to adjust the width of each column

\begin{paracol}{2} % Multi-column environment that automatically splits across pages

%---------------------------------------------------------------------------------
%	HEADER
%---------------------------------------------------------------------------------

\outputrecipeheader % Output the recipe header, automatically created using the information entered in the RECIPE INFORMATION block above

%---------------------------------------------------------------------------------
%	INGREDIENTS
%---------------------------------------------------------------------------------

\begin{recipeingredients}
	\ingredient{1 tbsp olive oil}
	\ingredient{20g butter}
	\ingredient{2 brown onions, halved,\\ finely chopped}
	\ingredient{2 garlic cloves, crushed}
	\ingredient{500g beef mince}
	\ingredient{145g (1/2 cup) tomato paste}
	\ingredient{250mL (1 cup) dry red wine}
	\ingredient{2 x 400g cans of diced tomatoes}
	\ingredient{1 tbsp dried oregano}
	\ingredient{3 dried bay leaves}
	\ingredient{Salt and freshly ground\\ black pepper}
	\ingredient{1/3 cup fresh parsley, loosely packed, coarsely chopped}
	\ingredient{375g dried thin spaghetti}
	\ingredient{80g parmesan, to serve}
\end{recipeingredients}

%---------------------------------------------------------------------------------
%	METHODS
%---------------------------------------------------------------------------------

\begin{recipemethods}
	\method{Heat oil and butter in a large saucepan over medium-high heat. Add onion and garlic and cook, stirring, for 3 minutes or until onion softens. Add the mince and cook, stirring with a wooden spoon to break up any lumps, for 5 minutes or until the mince changes color.}
	\method{Add the tomato paste, wine, tomato, oregano and bay leaves, and bring to the boil. Reduce heat to medium and simmer, stirring occasionally, for 1 hour or until sauce thickens. Taste and season with salt and pepper. Stir in the parsley.}
	\method{Meanwhile, cook the spaghetti in a large saucepan of salted boiling water following packet directions until al dente. Drain.}
	\method{Divide the spaghetti among bowls and spoon over bolognese sauce. Grate over the parmesan and serve immediately.}
\end{recipemethods}

%---------------------------------------------------------------------------------

\end{paracol}

\end{document}
